\chapter{The science of semantics: Aims, methods, and aspirations}
In this concluding chapter I link the discussion of internalist and externalist semantics with a discussion of what scientific explanations look like in general. I argue that a fruitful scientific explanation is one that aims to uncover the underlying mechanisms in virtue of which the observable phenomena are made possible, and that a scientific semantics should be doing just that. If this is so, then a science of semantics is unlikely to be an externalist one, for reasons having to do with the subject matter and form of externalist theories.

I should make clear at the outset that even though I argue that externalism is a hermeneutic project and thus not a scientific one, my criticism should not be taken to be dismissive of the hermeneutic approach to semantics. The hermeneutic approach has provided and continues to provide great insight into the use of language in human social interaction, but this book is concerned with what a science of semantics should look like. My criticism is thus not aimed at externalists per se but rather at those of them who claim to be part of the scientific project. Considering the latter type of externalists, one should of course judge their theories by the same standards as internalist theories (and scientific theories in general). In other words, taking for granted the claim by both sides to be doing science, the real interest in the externalism/internalism debate comes when one considers which questions, aims, and theoretical interests are more likely to produce a fruitful explanatory scientific semantics.


\section{The nature of scientific explanations}
What follow are some remarks on the nature of scientific explanation. I argue that the aims and practices of externalism are orthogonal to those of cognitive psychology. The main reason for this is that, unlike the externalist approach, a fruitful scientific explanation is one that aims to uncover the underlying mechanisms in virtue of which the observable phenomena are made possible. I first unpack this view of scientific explanation, and then offer some remarks on the implication of this view for externalism.

One can make the strong claim that many scientific explanatory theories – perhaps all but physics – follow what \citet{Thagard2012} refers to as the \textit{mechanista} view of scientific method, according to which to explain a phenomenon is to unearth the mechanism that produces it. \citet{Fodor1968} and \citet{Cummins1975,Cummins1983} are early versions of this sort of approach to explanation. It has been developed more recently by, among others, \citet{BechtelRichardson1993}, \citet{Glennan1996}, \citet{MachamerCraver2000}, \citet{Craver2007}, and \citet{Bechtel2009}, though this conception goes back to Descartes and Boyle (see \citealt{Bechtel2011} for discussion). On this view, the aim of science is the discovery of mechanisms rather than laws. \citet{MachamerCraver2000} argue that much of the practice of science can be understood in this way. Mechanisms, according to them, are “identified and individuated by the activities and entities that constitute them, by their start and finish conditions, and by their functional roles” \citep[6]{MachamerCraver2000}. A mechanism is defined by them as a regular series of activities of entities that bring about a particular phenomenon. The emphasis here is on what the activities of mechanisms produce, rather than merely on the changes in the properties of the mechanisms. The construal of scientific explanation in terms of the unearthing of mechanisms is a different project to that of the discovery of laws. In fact, subsumption under law is a misunderstanding of how fruitful scientific explanation works (see \citealt{Cummins2000}). \citet[8]{MachamerCraver2000} give an example from biology according to which if a single base were changed in DNA and the mechanism of protein synthesis operated as usual, then a counterfactual would be supported. “No philosophical work is done,” they say, “by positing some further thing, a law, that underwrites the productivity of activities". Activities are constitutive of mechanisms, and it is they that make phenomena intelligible. In other words, the intelligibility consists “in mechanisms being portrayed in terms of a field’s bottom out entities and activities” \citep[21]{MachamerCraver2000}. So it is not regularities or laws that explain. Rather, what does the explaining are the mechanisms in virtue of which the observed regularities are made possible. 

It should be noted that mechanistic explanations are not reductive explanations – one cannot use them to deductively predict from a lower level what will occur at a higher level. The decomposition into mechanisms (and into mechanisms of mechanisms) preserves the higher levels, and indeed a mechanistic explanation would be incomplete without a hierarchy of levels. In other words, multiple levels are required in order to properly explain a particular phenomenon, and it is the integration of different levels that makes phenomena intelligible. Moreover, it is striking that despite its apparent prevalence, mechanistic explanation was largely ignored in the philosophy of science in the twentieth century. \citet{Bechtel2009} shows how biologists and psychologists rarely make use of laws in giving explanations, and in the relatively few cases in which they do the laws tend to be those of physics or chemistry (see also \citealt{Bechtel2008,BechtelAbrahamsen2005}). In the case of biology, say, there is an “ubiquity of references to mechanism” and a “sparseness of references to laws” \citep[423]{BechtelAbrahamsen2005}. \citet[140]{Cummins2000} speaks of the scandal in regard to the widespread belief that scientific explanation is subsumption under law: “Laws tell us what the mind does, not how it does it. We want to know how the mind works, not just what it does”. He gives the example of the McGurk effect in psychology: “no one thinks that the McGurk effect explains the data it subsumes”, no one “would suppose that one could explain why someone hears a consonant like the speaking mouth appears to make by appeal to the McGurk effect”; this is because that “just \textit{is} the McGurk effect” \citep[119, emphasis in original]{Cummins2000}. In other words, laws describe the data, they do not explain the data. 

\citet{BechtelAbrahamsen2005} discuss how the identification of phenomena in biology precedes their explanation. There is a sense in which there is no other way to go about scientific explanation – we cannot know in advance what it is that needs explaining. Asking the right questions in science is an important part of what makes a particular explanatory theory successful. In cognitive psychology, identifying phenomena of, say, behavioural dispositions or of language use precedes their theoretical explanation. We of course need to know what it is that humans are doing when they use language, but that is not an explanation – it is a description. What we have in externalist theories is a description of regularities and “laws” of language use or of behaviour. There is much debate about ascription, about what a particular behaviour or linguistic output should be labelled as. But regardless of the value and interest of such descriptions – and it is far from nil – such theories are not fruitful \textit{explanatory} theories: they are the explananda, not the explanantia. To conceive of the externalist research project as scientific – specifically, as having the same aims as cognitive psychology and internalist semantics – is a misunderstanding that fails to see the force and value of hermeneutic research. Investigating the way in which words are individuated and the way in which social norms come into play when people use language, amongst other topics, is a valuable and interesting project. But this tells us little about the mechanisms in virtue of which such language use is made possible. Another way to put the matter is as follows. Hermeneutic research does not explain the underlying mechanisms of language but rather \textit{uses} them to investigate the world via language. Internalism, on the other hand, takes its subject matter to be the underlying mechanisms themselves, not their use.

Externalism is a hermeneutic project in which linguistic abilities are used in the investigation of the world. Its methods are ill-suited to the investigation of those underlying mechanisms of language use, for that leads to explanatory vacuousness in which the abilities that are purportedly being explained are relied upon implicitly. The next section is devoted to showing in detail that if externalism is understood as science then it indeed leads to such illegitimate reliance on the very phenomenon to be explained.

\section{Externalism and scientific explanations}
As noted above, there are a number of leading externalists who explicitly conceive their projects to be scientific ones. Horwich, for example, says of Davidson’s externalist truth-theoretic programme that it “became widely accepted, instigating several decades of ‘normal science’ in semantics” \citep[371]{Horwich2001}. Despite the immense popularity of truth-theoretic semantics, Horwich is one of a handful of critics within the externalist camp that have called into question the concept of truth as a basis for a semantic theory. Their deflationary theory of truth argues that to assert that a particular sentence is true is equivalent to merely asserting the sentence on its own. That is, the claim is that asserting the sentence “‘snow is white’ is true” is equivalent to merely asserting that snow is white. In other words, “[o]ur use and grasp of the concept of truth is adequately explained by our tendency […] to accept $<$\textit{A}$>$ \textit{is true} when, and only when, we are prepared to accept \textit{A}” \citep[12, emphasis in original]{Armour-GarbBeall2005}. Deflationists argue that in order to understand a concept one must consider its function, and therefore in order “to understand what truth \textit{is}, we must consider what truth \textit{does}” and “consideration of what it does reveals that it has no underlying nature or structure at all – there is nothing to truth” \citep[17, emphasis in original]{Armour-GarbBeall2005}. The deflationary account thus precludes the analysis of the nature of meaning in terms of truth conditions.

The deflationary alternative to truth-theoretic semantics, however, is still an externalist account of semantics. It is worth looking at this alternative, for it shows that the problems with truth-theoretical theories of meaning are due to their externalist conception of meaning and not due to their truth-theoretical formal apparatus. Horwich’s use-based semantics, whilst not truth-theoretical, is still externalist and suffers from similar problems if construed as a cognitive psychological scientific project. Horwich argues that his use-based semantics, essentially a defence of Wittgenstein’s idea of a use theory of meaning, is compatible with a linguistics construed as an empirical science, but the reasons for rejecting this claim are the same as the reasons for rejecting any externalist theory of meaning if one wishes to construct a fruitful scientific semantics.

Horwich is critical of mainstream formal semantics and argues that “as far as explaining our linguistic activity is concerned, there is no reason at all to think that understanding has a truth theoretic basis” \citep[318]{Horwich2008}. He claims that while the problems truth-theoretic semantics presents “are highly challenging, requiring considerable skill and ingenuity, and that enormous progress has been made in these endeavours over the last forty years or so”, citing “such progress is not enough to vindicate truth-theoretic semantics as an \textit{empirical} subject, as an integral part of the global scientific enterprise” \citep[318, fn. 12, emphasis in original]{Horwich2008}. He argues that in order to be a part of science, truth-theoretic semantics must show how their derivations have contributed to the explanation of observable events. However, “that has not, and cannot, be done” \citep[318]{Horwich2008}. His main objection has to do with compositionality and the assumption of formal semanticists that “the project of semantics needs to start […] with theoretical assumptions about \textit{the meanings of sentences}” \citep[314, emphasis in original]{Horwich2008}. As we saw above, truth-theoretic semantics is an analysis that focuses on sentences. As \citet[4]{LeporeLudwig2007} put it, the “goal is not to provide an analysis of the concept of meaning, or an analysis piecemeal of particular words or what it is for someone to understand them, but to illuminate as a whole the set of concepts deployed in understanding other speakers by considering how one could confirm such a theory on the basis of evidence described without appeal to those concepts”. Horwich takes the opposite approach, for he believes that compositionality is relatively easy to accommodate and thus one needs to first “somehow identify […] the theoretical-meanings of \textit{words}, and then, presupposing compositionality, to trivially deduce the theoretical-meanings of sentences” \citep[314, emphasis in original]{Horwich2008}.

Inverting the focus of semantics from sentences to words, says Horwich, has the effect of nullifying truth-theoretic semantics, for truth conditions apply to sentences and cannot apply to words. Given this focus on words, Horwich argues that the theoretical characterisation of the meanings of words will be deduced from “certain facts concerning sentence \textit{usage}, rather than sentence \textit{meaning}” \citep[314, emphasis in original]{Horwich2008}. That is, once the meanings of words are deduced from observations of sentence usage, “we will – in light of compositionality – be able to arrive […] at the meanings of sentences”. Thus, according to Horwich’s use-based semantics and \textit{pace} Davidson, “it will be trivially easy to deduce what any sentence means from the structure of that sentence and what its words mean” \citep[314]{Horwich2008}. A simple example of Horwich’s use-based semantics is the following:
\begin{quote}
Presumably our understanding of the sentence “dogs bark” arises somehow from our understanding of its components and our appreciation of how they are combined. That is to say, “dogs bark” somehow gets its meaning (or, at least, one of its meanings) from the meanings of the two words “dog” and “bark”, from the meaning of the generalization schema “ns v”, and from the fact that the sentence results from placing those words in that schema in a certain order. \citep[154]{Horwich1998}
\end{quote}
So on this account the meaning of the sentence \textit{dogs bark} is deduced by combining a word meaning \textit{dog} with a word meaning \textit{barks}. Therefore, according to Horwich, understanding complex expressions is nothing over and above understanding their parts and knowing how they are combined. This is what he means when he claims that compositionality is a relatively trivial matter. Knowing the meanings of words and being aware of their mode of combination is all that is required, says Horwich, in order to understand the meanings of sentences: “No further work is required; no further process needs to be involved, leading from those initial conditions to the state of understanding the sentence” \citep[155]{Horwich1998}.

It might seem that Horwich’s account is compatible with an internalist semantics, for the latter also claims that all that is required for the explanation of sentence meaning is the primitive lexical elements and the syntax defining the ways in which they can be combined. But the way in which the primitives are explained in internalist semantics is very different to Horwich’s use-based account. Horwich is critical of truth-theoretic semantics because he thinks it lacks the necessary explanatory power. That is, he says, the observable events that are of interest to semantics are “items of verbal activity – both mental and behavioural”, and semantics “is obliged to explain […] facts concerning the circumstances in which sentences are \textit{accepted}” \citep[315, emphasis in original]{Horwich2008}. But truth-theoretic semantics cannot sufficiently explain such facts and is thus a failed enterprise: it cannot (but it must, says Horwich) “tell us what it is about, e.g., ‘The sky is blue’ that explains why it tends to be \textit{recognized} as true if and only if it \textit{is} true” \citep[317, emphasis in original]{Horwich2008}. 

This is clearly still an externalist semantic theory, for even though it rejects truth conditions it claims that “the underlying basis of each word’s meaning is the (idealized) law governing its usage -- a law that dictates the ‘acceptance conditions’ of certain specified sentences containing it” \citep[26]{Horwich2005}. This law of acceptance conditions, argues Horwich, solves the puzzle of why it is that \textit{The sky is blue} tends to be recognised as true. The law would stipulate, for example, that the meaning of \textit{red} “stems from the fact that its law of use is a propensity to accept ‘That is red’ in response to the sort of visual experience normally provoked by observing a clearly red surface” or that “‘and’ means what it does because the fundamental regularity in its use is our acceptance of the two-way argument schema, ‘p, q // p and q’” \citep[26]{Horwich2005}. The law of acceptance conditions, which is supposed to underwrite Horwich’s semantic theory, is explicitly understood to be on par with a linguistics construed as an empirical science. But as we’ll now see, laws of this kind are problematic at best.

Horwich argues that the phenomena that semantics needs to explain are those of sentence acceptance. He elaborates: “I don’t mean ‘accepted as \textit{grammatical}’, but ‘accepted as \textit{true}’, i.e., ‘in the belief-box’. Acceptance sometimes leads to \textit{utterance} (depending on the speaker’s desires); therefore explaining the acceptance of a sentence may contribute to explain its being uttered” \citep[315, fn. 9, emphasis in original]{Horwich2008}. Sentence acceptance is explained by the following:
\begin{quote}
The meaning of a word, w, is engendered by the non-semantic feature of w that explains w’s overall deployment. And this will be an acceptance-property of the following form:— “that such-and-such w-sentences are regularly accepted in such-and-such circumstances” is the idealized law governing w’s use is [\textit{sic}] (by the relevant “experts”, given certain meanings attached to various other words). \citep[28]{Horwich2005}
\end{quote}
According to the use theory of meaning, then, a word means what it does “\textit{in virtue} of its basic use; a word’s use is \textit{responsible} for its meaning what it does. Thus, not only does a meaning-property \textit{supervene} on a basic acceptance property, but possession of the former is \textit{immediately explained} by possession of the latter” \citep[32, emphasis in original]{Horwich2005}.

Horwich argues that insofar as “linguistics is an empirical science – standing alongside psychology, neurology, biology, physics, etc.”, then such acceptance-laws “should be testable against concrete observable events” \citep[315]{Horwich2008}. Thus, “the semanticist of a given language ought to be looking, concerning each word, for the basic law governing its use” \citep[319]{Horwich2008}, and if such laws are forthcoming and explanatorily fruitful, Horwich believes that “[s]emantics would then somewhat resemble fundamental physics” \citep[318]{Horwich2008}. The phenomena of sentence acceptance is supposed to cohere with phonology, syntax, and pragmatics to yield a science of language use. Horwich argues that truth-theoretic semantics cannot yield such a science but that his use-based semantics can. However, both are problematic if construed as science. One problem with sentence acceptance, a main tenet of Horwich’s theory, is that it is unclear whether it can be generalised beyond the examples that Horwich gives. As \citet{Schiffer2000} argues, meaning-constituting properties are supposed to be acceptance properties, but it is not even clear whether relatively simple words like \textit{dog} have acceptance properties. As he puts it, there are no plausible candidates for “a kind of ‘dog’-containing sentence \textit{K} and a kind of circumstance \textit{C} such that a speaker for whom ‘dog’ means \textsc{dog} will be disposed to accept a sentence of kind \textit{K} in circumstances of kind \textit{C}, and that fact will belong to the explanation of his accepting any other sentence that contains ‘dog’” \citep[534]{Schiffer2000}. For instance, “‘[d]og’ may mean \textsc{dog} for someone who is blind or who does not know what a dog looks like, so it cannot be required that anyone who understands ‘dog’ must be disposed to accept ‘That’s a dog’ when confronted with a paradigm dog” \citep[534]{Schiffer2000}.

Even granting the validity of acceptance properties, it is unclear whether sentence meaning can be reduced to sentence acceptance because the latter involves much more than what is traditionally thought of as sentence meaning. Consider the following example from \citet{Gupta1993}. Suppose that a predicate G has the following stipulative definition: a thing is G if and only if it is red and round. Given the close connection between the acceptance properties and meaning properties of G, it may appear that the two can be regarded as the same thing. But such a definition, argues Gupta, may play only a minimal role in the explanation of a person’s acceptance of sentences containing G. The fundamental role in such an explanation may be played by, for example, the authority of some expert (if, say, the person trusts the expert’s colour reports). There is thus “little reason to think […] that ‘explanatorily basic patterns [of sentence acceptance]’” in Horwich’s use theory of meaning “provide the meaning of a word”, for “plainly, the acceptance of sentences depends not just on the meanings of words but also on the methods of obtaining information (and misinformation) about the world”. In other words, “we should distinguish general ideas such as ‘meaning is use’ and ‘meaning explains use’ from Horwich's particular claim. The former may express truisms, the latter does not” \citep[666]{Gupta1993} (see also \citealt{Gupta2003}).

Gupta’s remark that the acceptance of sentences depends not just on the meanings of words but also on the methods of obtaining information about the world hints at the main reason why externalist theories such as Horwich’s cannot serve as a foundation upon which to construct a science of semantics: such theories have a problematic choice of subject matter. The scope of semantic theories was discussed by \citet{KatzFodor1963} in the early days of internalist semantics and it’s worth briefly revisiting here, for it bears directly on the problems with the scope of externalist theories of meaning. Katz and Fodor ask the reader to compare the following three sentences: \textit{Should we take junior back to the zoo? Should we take the lion back to the zoo? Should we take the bus back to the zoo?} They then remark that, for example, “[i]nformation which figures in the choice of the correct readings for these sentences includes the fact that lions, but not children and busses, are often kept in cages”. After listing a handful of other examples, they note that the “reader will find it an easy matter to construct an ambiguous sentence whose resolution requires the representation of practically any item of information about the world he chooses”. But “a complete theory of setting selection must represent as part of the setting of an utterance any and every feature of the world which speakers need in order to determine the preferred reading of that utterance”, and “practically any item of information about the world is essential to some disambiguations” \citep[179]{KatzFodor1963}. If this is so then a number of conclusions follow.

The first conclusion is that a theory that insists (as externalism does) on including the mind’s relations to the external world in a theory of language cannot hope to find reliable relations of this sort (let alone systematising them into a fruitful explanatory theory). Second, as Katz and Fodor note, “such a theory cannot in principle distinguish between the speaker's knowledge of his language and his knowledge of the world, because, according to such a theory, part of the characterization of a \textsc{linguistic} ability is a representation of virtually all knowledge about the world that speakers share” \citep[179, emphasis in original]{KatzFodor1963}. Thirdly, Katz and Fodor remark that “since there is no serious possibility of systematizing all the knowledge of the world that speakers share, and since a theory of the kind we have been discussing requires such a systematization, it is ipso facto not a serious model for semantics” \citep[179]{KatzFodor1963}. The same is true of externalism. Moreover, despite the efforts of Horwich and others, due to the creative aspect of language use there is little chance of constructing a science of language use \citep{Chomsky1966,McGilvray2001,McGilvray2005,Asoulin2013}.

But there is a deeper reason. The laws of language use, if they can be formulated at all, at best tell us what a language user \textit{does}, they do not tell us why that is the case nor explain the underlying ability to do so. It is the latter that science seeks to uncover. Scientific laws \textit{describe} the data in question, not explain the data. As \citet{Cummins2000} discusses, there is now a consensus that the deductive nomological (DN) sense in which laws are explanatory is a myth and that “the suspicion grows that it \textit{cannot} be done successfully” because there is no difference between laws and data: “No laws are explanatory in the sense required by DN” \citep[119, emphasis in original]{Cummins2000}. \citet[120]{Cummins2000} quips that in psychology “we are overwhelmed with things to explain, and somewhat underwhelmed by things to explain them with”. The same is true of theories of meaning that take an externalist approach but still claim to be psychological or non-hermeneutic. We are indeed overwhelmed with things to explain in externalism: phenomena of meaning ascription, sentence acceptance properties, truth-evaluable judgements and the like are fascinating phenomena of language use. But they are data of language use, not scientific explanations of language use. This conflation, according to Cummins, “derives from a deep-rooted uncertainty about what it would take to really explain a psychological effect” \citep[121]{Cummins2000}. 

We saw above that Horwich claims that if the science of semantics is done the way he proposes then semantics would "somewhat resemble fundamental physics” \citep[318]{Horwich2008}, but this reflects the very conflation that Cummins points to. Let us see why. Semantics cannot resemble fundamental physics any more than geology can, for they are what \citet{Fodor1974} famously called special sciences. Unlike fundamental physics, the special sciences do not yield general laws of nature but only “laws governing the special sorts of systems that are their proper objects of study”. Laws of psychology are laws in situ, which “specify effects—regular behavioral patterns characteristic of a specific kind of mechanism” \citep[121]{Cummins2000}. But notice the crucial difference here: the laws in question \textit{describe the effects} of the specific kind of mechanism which is their subject matter. But in order to move from a description to an explanation we need an account of the mechanism itself. Notice that this is not the case at the level of fundamental physics, where “laws are what you get because, at a \textit{fundamental} level, all you can do is say how things are”. That is, the “things that obey the fundamental laws of motion (everything) do not have some special constitution or organization that accounts for the fact that they obey those laws” because the “laws of motion just say what motion is in this possible world” \citep[122, emphasis in original]{Cummins2000}. As Cummins argues (\citealt{Cummins1975,Cummins1983,Cummins2010,RothCummins2014}), special sciences like psychology (and, of course, like a semantics construed as science) “should seek to discover and specify the effects characteristic of the systems that constitute their proprietary domains, and to explain those effects in terms of the \textit{structure} of those systems, that is, in terms of their constituents (either physical or functional) and their mode of organization” \citep[122, emphasis in original]{Cummins2000}.

I would like to briefly return to the discussion of Davidon’s truth-theoretic semantics to show that the problematic nature of the explanations in use-based semantics \textit{qua} scientific explanations is also present in the Davidsonian programme (and, indeed, in any externalist theory of meaning of this sort). Lepore and Ludwig remark that the “centrepiece and nexus of Davidson’s philosophy” is the project of the radical interpreter, and that “the stance of the radical interpreter of another speaker [… is] methodologically basic in understanding language and connected matters” \citep[viii, 2]{LeporeLudwig2005}. This stance stems from the problem confronting linguists constructing a grammar of a language they do not yet understand. The problem is how to choose amongst competing theories of meaning for the language under investigation. In other words, “given a theory that would make interpretation possible, what evidence plausibly available to a potential interpreter would support the theory to a reasonable degree?” \citep[125]{Davidson1973}. Davidson claims that the problem of interpretation is “domestic as well as foreign: it surfaces for speakers of the same language in the form of the question, how can it be determined that the language is the same?” \citep[125]{Davidson1973}. That is, even in cases of everyday communication by speakers of the same language, the speakers are in a sense theorising interpreters. Two speakers of English, say, who successfully communicate to each other are thus each possessors of a theory of interpretation. 

Davidson takes his theory of meaning to model what interpreters are doing in this form of theorising, and he thus uses the terms \textit{interpretation} and \textit{understanding} as if they were interchangeable (see \citealt{Mulhall1987}). A meaning theory, says Davidson, must allow the interpreter “to understand any of the infinity of sentences the speaker might utter” in the language \citep[127]{Davidson1973}, and thus “someone who knows the theory can interpret the utterances to which the theory applies” \citep[128]{Davidson1973}. The theory Davidson has in mind, of course, is a truth-theoretic theory of meaning, and thus the project of the radical interpreter becomes that of “confirming a truth theory for a speaker’s language that can be used to interpret correctly the speaker’s utterances” \citep[3]{LeporeLudwig2005}.

As already noted, this is a conflation of understanding as speaker and understanding as theorist. The project of the radical interpreter that aims at constructing a compositional meaning theory for a natural language is a hermeneutic project of interpretation because it takes for granted the underlying mechanisms of language. If we take scientific explanations as seeking to unearth the mechanisms responsible for the observed phenomena, then truth-theoretic semantics leaves unexplained the abilities it purports to explain. Davidson’s truth-theoretic semantics is a case in point in regard to an externalist approach being ill suited to a scientific semantics. The conflation of the two senses of understanding arises in part from the often implicit assumption that the investigation of the underlying mechanisms will not be philosophically illuminating. As Davidson notes in regard to his theory: “The point of the theory was not to describe how we actually interpret, but to speculate on what it is about thought and language that makes them interpretable”. Furthermore, he remarks that “[a]ll that is lacking at the start [of constructing a theory of meaning] is a shared language and prior knowledge of each other’s attitudes”. Therefore, since “the theory and the official story of how it can be applied are already remote from actual practice, we must expect that the theory will throw only the most oblique light on the acquisition of a first language, and less still on the origins of speech” \citep[128]{Davidson1995}. Such statements clearly distinguish Davidson’s project from the internalist (and biolinguistic) project. It is clear that Davidson does not attempt to explain the mechanisms that underlie language production and comprehension, for his theory explicitly assumes a shared language and prior knowledge of other people’s propositional attitudes. 

The quest to construct an interpretive truth theory is seen by Davidsonians as the distinguishing mark of a semantic theory: this sets the subject area of their theory apart from an internalist theory of meaning that seeks to explain the very abilities that externalist theories use in constructing their own theories. Davidson is explicit about this:
\begin{quote}
I want to know what it is about propositional thought – our beliefs, desires, intentions, and speech – that makes them intelligible to others. This is a question about the nature of thought and meaning which cannot be answered by discovering neural mechanisms, studying the evolution of the brain, or finding evidence that explain the incredible ease and rapidity with which we come to have a first language. \citep[133]{Davidson1995}
\end{quote}
In other words, the nature of the underlying mechanisms by which language is acquired, produced, and comprehended in the heads of speakers cannot be unearthed via the radical interpreter project. If this is so, then Davidson’s claims that his project is scientific displays the conflation between the two senses of understanding (\textit{verstehen} and \textit{erklären}).

Another way to understand the conflation is by considering the difference between a traditional grammar and a generative grammar. This is the distinction between, on the one hand, a descriptive or interpretive theory of language and, on the other hand, an explanatory theory of language \textit{qua} science. The latter is a grammar in the I-language sense, which is an account of the ideal speaker/hearer’s competence. Furthermore, “[i]f the grammar is […] perfectly explicit – in other words, if it does not rely on the intelligence of the understanding reader but rather provides an explicit analysis of his contribution – we may (somewhat redundantly) call it a \textit{generative grammar}” \citep[4, emphasis in original]{Chomsky1965}. Recall that a generative grammar is understood intensionally, meaning that it focuses on the specific procedure encoded in the mind that generates the strings of the language (as opposed to the E-language conception that focusses on the strings themselves). A generative grammar is distinct from both an E-language grammar and a traditional prescriptive grammar. Traditional grammars describe how a particular language is or should be used. However, as \citet[237]{Chomsky1980} remarks, traditional grammars “do not provide an analysis of the qualities of intelligence that the reader brings to bear on the information presented”. The same is true in the case of externalist theories such as truth-theoretic semantics. Traditional descriptive grammars and interpretive truth-theoretic theories of meaning, whatever their merits (and one should not be dismissive of their accomplishments), provide only examples and hints as to the underlying nature of the language. That is, the success of traditional grammars (and the success of externalist semantics) rests entirely on their pairing with “an intelligent and comprehending reader” \citep[528]{Chomsky1962}. Indeed, the remark of \citet[529]{Chomsky1962} almost 60 years ago that “[r]eliance on the reader’s intelligence is so commonplace that its significance may easily be overlooked”, is still pertinent today in the case of externalist theories of meaning.

My criticism of descriptive or interpretive theories of language does not stem from the claim that they omit certain facts. Considering their subject matter, such theories haven’t left out the mechanisms in the head in virtue of which language is made possible, for that is not their aim. From the perspective of scientific explanations, however, they give an incomplete picture of the nature of language because they assume, indeed they build upon and would be unusable without, the abilities of language users. It is this ability on which internalism and biolinguistics would like to shed light. \citet[11]{LeporeLudwig2005} appear to concur when they remark that “Davidson treats compositional meaning theories as empirical theories, theories of particular speakers or natural languages, which must be confirmed on the basis of public evidence”. These theories belong “in the context of a theory of interpretation of human action in general”, and thus “from this perspective, the role of a theory of interpretation is to identify and systematize patterns in the behaviour of speakers in relation to their environment”.

\begin{center}
***
\end{center}

\citet[21]{Fodor2000a} notes approvingly in regard to the modern form of the representational theory of mind that it has “shed a feature that traditional versions of the doctrine, rationalist and empiricist, invariably took for granted: that RTP [representational theory of perception] must provide not just a psychology of perception but an epistemology, too”. Fodor distinguishes what psychology does, which is, among others, to explain the mechanisms that underpin belief fixation, from what epistemology does, which is, among others, to investigate whether one is justified in having a particular belief or meaning, or whether one’s particular representation of, say, an object in the world, is a true representation or a misrepresentation. He refers to the attempt to incorporate within one theory both how one has a particular visual perception and how one is justified in believing that particular visual perception as a “double burden” and as the core of what was wrong with the traditional representational theories of perception. However, Fodor fails to see that this same double burden is carried by externalist theories of meaning such as his own.
	
A different double burden in regard to theories of meaning is discussed by \citet{Lepore1983,Lepore1983a}, who argues that the problem of how to characterise what is in the speaker’s head is a “non-issue” \citep[185, fn. 5]{Lepore1983}. That is, the problem of unearthing the underlying mechanisms of language in the head “arises only if, as many do, one views semantics as a subfield of psychology”. In other words, “[i]f we assume that questions about knowledge and understanding of language are psychological questions, then semantics should be a subfield of psychology”. Lepore disagrees with this conception of semantics and argues for an externalist conception according to which semantics “properly understood, is not a subfield of psychology but of epistemology”. Thus, he concludes, since semantics and questions about knowledge and understanding of language are not psychological questions but are instead epistemological questions, “we need not worry about what’s in the speaker’s head – whatever that may mean” \citep[185, fn. 5]{Lepore1983}. If the goals of a theory of meaning are understood as part of epistemology, then perhaps one can make the case that psychology is beside the point. But if this is the case then what is one to make of the claims of Burge, Davidson, Fodor, and Horwich that their externalist theories of meaning are scientific? In other words, if psychology aims to discover the mechanisms in virtue of which language is made possible, and if that means that psychology is part of science, then, \textit{pace} Lepore, if one’s research programme is scientific then we do need to worry about what’s in the head, we do need to have a semantics that is informed by (and in turn informs) cognitive psychology.

Either an externalist theory of meaning is scientific and should thus be answerable to or at least in principle be able to be integrated with the other sciences, or it is a hermeneutic theory and thus, as Lepore argues, it isn’t answerable to and can remain impartial in regard to science. I argued in this book that externalist theories are indeed hermeneutic and thus are not explanatory theories in the scientific sense. 